\section{Capítulo 5: Estimación de Costos}

Es necesario conocer lo más posible sobre el proyecto para confeccionar el presupuesto. Por eso, la definición del proyecto debe ser lo mas completa posible. En esta se incluyem:

\begin{itemize}
    \item Antecedentes de la zona y costos de referencia: visita a terreno, mercado de trabajadores especializados, fuentes de abastecimiento.
    \item Antecedentes disponibles en oficina: n° operarios, personal, cantidad de equipos, valores de sueldos, rendimientos, costos, etc.
\end{itemize}

\subsection{Estimación conceptual de costos}

\begin{itemize}
    \item Estimación preliminar: parte del análisis de factibilidad económica de un proyecto.
    \item Estimación conceptual: dueño del proyecto la utiliza para decisiones tempranas o su propio presupuesto.
    \item Estimación detallada: basada en mediciones de ctdes, una vez que el diseño del proyecto esta detallado.
    \item Estimación definitiva: actualización de la estimación detallada, con precios actualizados.
\end{itemize}

La importancia entonces de la estimación conceptual es que, si ésta se realiza en forma adecuada, permite disminuir los costos finales de un proyecto.

\begin{figure}[H]
    \centering
    \begin{subfigure}[b]{0.45\textwidth}
        \centering
        \includegraphics[width=\textwidth]{FOTOS/est_costos.png} % Reemplaza "imagen1.jpg" por el nombre de tu archivo
        \caption{Proceso de estimación de costos}
        \label{fig:imagen1}
    \end{subfigure}
    \hfill
    \begin{subfigure}[b]{0.45\textwidth}
        \centering
        \includegraphics[width=\textwidth]{FOTOS/sub_costo.png} % Reemplaza "imagen2.jpg" por el nombre de tu archivo
        \caption{Importancia de la estimación}
        \label{fig:imagen2}
    \end{subfigure}
    \label{fig:imagenes_juntas}
\end{figure}

\subsubsection{Métodos para la estimación}

\begin{itemize}
    \item Paramétrica: Se basa en relaciones empíricas entre costos y parámetros de desempeño (tamaño, calidad, complejidad), considerando el método de construcción. Esto se conoce como CER (Cost Estimating Relationship).
    \item Factores: Aplica un factor a un ítem clave de costo para estimar otros ítems necesarios, basado en datos históricos.
    \item Modelo de costos: Utiliza un modelo estándar del proyecto que, procesando datos de costo, estima el costo total.
    \item Probabilística: Similar al PERT, incorpora riesgo mediante estimaciones múltiples (optimista, probable, pesimista) y evalúa la incertidumbre proporcional a la varianza.
\end{itemize}

\subsection{Estudio detallado de un presupuesto}

Un presupuesto total de una obra, o presupuesto de venta, es la estimación total que el ejecutor cobra al mandante. Incluye:

\begin{itemize}
    \item Presupuesto de diseño: Costo del diseño del proyecto, abarcando arquitectura, cálculo, instalaciones, etc.
    \item Costos directos: Estimación del costo de materiales, mano de obra y equipos directamente relacionados con actividades específicas.
    \item Gastos generales: Costos directos no imputables a una actividad específica, prorrateados en diferentes partidas (ej. salarios administrativos, consumibles).
    \item Gastos generales indirectos: 
    \begin{itemize}
        \item Costo financiero: Recursos necesarios para ejecutar las obras mientras se reciben los pagos o ingresos esperados.
        \begin{itemize}
            \item Garantías: Costos de las boletas de garantía requeridas en los contratos (ej. garantía de seriedad de la propuesta, buena ejecución).
            \item Gastos generales de oficina central: Contribución de la obra a la administración central, necesarios incluso sin obras (ej. sueldo gerente, oficinas, teléfonos).
        \end{itemize}
        \item Imprevistos: Riesgo de gastos no controlables. Disminuyen con mayor seguridad en precios y cantidades.
        \item Utilidad: Ganancia estimada, generalmente calculada como porcentaje del presupuesto del proyecto, dependiendo de rentabilidad, complejidad y riesgo.
        \item Impuestos: En Chile, las obras de construcción están sujetas al IVA. Las viviendas para compradores finales tienen un crédito del 65\%, reduciendo el IVA efectivo al 6,65\% (cuando el IVA es 19\%). Desde 2008, esta franquicia se limita a viviendas de hasta 3.000 UF y desaparece para aquellas de 4.500 UF.
    \end{itemize}
\end{itemize}

\subsubsection{Etapas para el estudio de un presupuesto}

\begin{enumerate}
    \item \textbf{Calendario de licitación}: Configuración de fechas clave para el análisis de precios, recepción de cotizaciones y revisión de ofertas.
    \item \textbf{Método constructivo}: Definición del método de construcción que determina el costo.
    \item \textbf{Estrategia presupuestaria}: Consideración de carga de trabajo, competidores, y tipo de proyecto.
    \item \textbf{Estudio de bases de licitación}: Análisis exhaustivo de plazos y condiciones impuestas por el mandante.
    \item \textbf{Cotizaciones}: Preparación de listas y solicitud de cotizaciones con especificaciones claras.
    \item \textbf{Subcontratos y cubicaciones}: Definición de prioridades en cubicaciones y análisis de contratos por precio unitario.
    \item \textbf{Estimación preliminar}: Realización de una estimación general del monto total.
    \item \textbf{Visita al terreno}: Evaluación in situ de las condiciones para la ejecución del proyecto.
    \item \textbf{Programación tentativa}: Elaboración de programas preliminares para optimizar recursos.
    \item \textbf{Revisión de precios locales}: Actualización de precios de mano de obra y maquinaria según condiciones locales.
    \item \textbf{Análisis detallado}: Estudio separado de costos directos, indirectos, gastos generales e imprevistos.
    \item \textbf{Revisión final}: Ajuste y verificación del presupuesto antes de la presentación de la oferta.
\end{enumerate}

\subsubsection{Estudio de costo directo}

Una vez aceptado el presupuesto, la omisión de algún ítem es una pérdida para el contratista. Las partidas deben ser medibles, presupuestables y controlables para cuantificar avances y comparar el progreso real con lo programado. Es recomendable identificar cada partida con un código y descripción, siguiendo la norma NCh 1156 Of. 99.

El segundo paso es definir la unidad de medida de cada partida, basada en las especificaciones técnicas o en la norma NCh 353 Of 2000. El tercer paso es cubicar las partidas, calculando las cantidades necesarias en volumen, área o longitud según la norma NCh 353 Of 2000.

El cuarto paso es estimar el costo de la partida mediante un análisis de precios unitarios. El costo directo o precio unitario (P.U.) debe incluir todos los costos directos necesarios para ejecutar el trabajo y ser compatible con las bases de medición. Se compone de cuatro elementos clave:

\begin{itemize}
    \item \textbf{Mano de obra}: Costo del personal involucrado, según especialidad y productividad.
    \item \textbf{Materiales}: Costo de los materiales en obra, basado en la cubicación y especificaciones técnicas.
    \item \textbf{Maquinaria y equipos}: Costo de equipos y herramientas, determinado por la planificación y estrategia de la obra.
    \item \textbf{Otros costos}: Herramientas y elementos menores que se requieren para una faena.
\end{itemize}

\subsubsection{Costo base de la mano de obra}

El costo de la mano de obra varía según las especialidades involucradas en un proyecto (profesionales, técnicos, maestros, ayudantes, administrativos, etc.). Factores que afectan su variabilidad incluyen:

\begin{itemize}
    \item Exigencia de habilidades especiales.
    \item Exigencia de conocimientos específicos.
    \item Exigencia de condiciones físicas especiales.
    \item Demanda de mano de obra en el mercado.
\end{itemize}

Para evaluar correctamente el costo, es importante estimar el rendimiento del trabajador, considerando que su tiempo en obra se divide en:

\begin{itemize}
    \item \textbf{Trabajo productivo} (60\%): Aporta directamente a la producción (e.g., colocación de moldajes).
    \item \textbf{Trabajo contributorio} (25\%): Actividades de apoyo necesarias (e.g., limpieza de superficies).
    \item \textbf{Trabajo no contributorio} (15\%): Acciones no productivas (e.g., espera por recursos).
\end{itemize}

Definiciones importantes:

\begin{itemize}
    \item \textbf{Remuneración (R)}: Pagos en dinero o en especies que recibe el trabajador por contrato.
    \item \textbf{Remuneración imponible (Ri)}: Parte de la remuneración usada para calcular imposiciones.
    \item \textbf{Sueldo}: Salario fijo determinado por contrato, pagado diaria o mensualmente. El sueldo líquido es el monto después de impuestos.
    \item \textbf{Pago por semana corrida}: Derecho a remuneración por domingos y festivos, condicionado a la asistencia semanal.
    \item \textbf{Gratificación}: Recompensa basada en utilidades de la empresa o un monto fijo anual de 4,75 salarios mínimos.
    \item \textbf{Feriado legal}: Derecho a 15 días hábiles de vacaciones anuales con remuneración íntegra.
    \item \textbf{Imposiciones}: Porcentajes sobre Ri destinados a fondos de pensiones y salud.
\end{itemize}

\noindent Los siguientes pagos no constituyen remuneración:
\begin{itemize}
    \item Asignación de movilización.
    \item Asignación por colación.
    \item Asignación por pérdidas de caja.
    \item Asignación por desgaste de herramientas.
    \item Viáticos (traslados, alojamiento y comidas con rendición).
    \item Prestaciones familiares por ley.
    \item Devolución de gastos por trabajo.
\end{itemize}

\textbf{El Trato}

Consiste en un convenio entre el empleador y el trabajador, o entre la empresa constructora y el subcontratista, según sea el caso, mediante el cual se fija un monto de dinero por realizar una faena específica en un plazo determinado.

\begin{equation}
    \text{TRATO} = (\text{Sueldo Base}) + (\text{Cantidad efectivamente realizada} \times \text{Precio de la unidad})
\end{equation}

Se utiliza para pagar faenas que son repetitivas o de un volumen importante en la construcción.

Para los convenios directos entre empleador y trabajador se utiliza directamente la fórmula anterior, en cambio en el caso de los subcontratos sólo se utiliza el segundo término de la ecuación.

La principal ventaja de utilizar el trato es obtener una mayor productividad en la obra.

La principal desventaja que este sistema de pago presenta, es que en ciertas ocasiones los trabajadores o los subcontratistas se despreocupan de la calidad requerida

\textbf{Componentes del costo de la mano de obra}

El costo de un trabajador incluye un costo fijo, un costo variable, un costo adicional por leyes sociales y costos asociados a gastos generales de faena.

\begin{itemize}
    \item \textbf{Costo fijo}: Incluye la remuneración del trabajador, considerando vacaciones, derecho a semana corrida e imposiciones. También puede incluir gratificaciones pagadas mensualmente. En construcción, considerando una jornada de lunes a sábado, se trabajan aproximadamente:
    \begin{equation}
        \text{Horas anuales trabajadas} = 300 \, \text{días} \times 8 \, \text{horas/día} = 2400 \, \text{horas}
    \end{equation}
    Además, debe sumarse el pago de los días domingos, en caso de que el contrato sea por día, según el Código del Trabajo. El ingreso anual por domingos es:
    \begin{equation}
    \text{Ingreso por domingos} = 53 \, \text{domingos} \times 8 \, \text{horas} \times 1.100 = 466.400 \, \text{pesos}
    \end{equation}
    \item \textbf{Costo variable}: 
    \begin{itemize}
        \item \textbf{Costos variables mensuales}:
        \begin{itemize}
            \item \textbf{Sobretiempo}: Recargo del 50\% en días hábiles y 100\% en domingos y festivos.
            \item \textbf{Trato}: Mayor costo dependiendo del coeficiente de trato.
            \item \textbf{Participaciones de producción}: Según se considere en la obra.
        \end{itemize}
        
        \item \textbf{Costo variable anual}:
        \begin{itemize}
            \item Gratificaciones y participaciones de producción anual, o 4,75 ingresos mínimos.
        \end{itemize}
    \end{itemize}
    \item \textbf{Costo adicional CAT}:
    \begin{itemize}
        \item \textbf{Seguro de accidentes}: Un 3\% sobre el total ganado.
        \item \textbf{Seguro de desempleo}: Contribuciones del 0,6\% del trabajador y 2,4\% del empleador.
        \item \textbf{Aporte patronal}: Corporación Habitacional (0,9\%) y Servicio Médico (2,1\%).
    \end{itemize}
    \item \textbf{Asignaciones}:
    \begin{itemize}
        \item \textbf{Asignación de alimentación}:
        \begin{equation}
        \text{Asignación de alimentación} = 300 \, \text{días} \times 200 = 60.000 \, \text{pesos}
        \end{equation}
        \begin{equation}
        \text{Asignación de alimentación (\%)} = \frac{60.000}{3.106.400} = 1,9\%
        \end{equation}
        
        \item \textbf{Asignación de movilización}:
        \begin{equation}
        \text{Asignación de movilización} = 300 \, \text{días} \times 600 = 180.000 \, \text{pesos}
        \end{equation}
        \begin{equation}
        \text{Asignación de movilización (\%)} = \frac{180.000}{3.106.400} = 5,8\%
        \end{equation}
        
        \item \textbf{Asignación por desgaste de herramientas}:
        \begin{equation}
        \text{Asignación por herramientas} = 300 \, \text{días} \times 400 \times 0,4 = 48.000 \, \text{pesos}
        \end{equation}
        \begin{equation}
        \text{Asignación por herramientas (\%)} = \frac{48.000}{3.106.400} = 1,5\%
        \end{equation}
    \end{itemize}
    \item \textbf{Indemnizaciones}: Costos en los que se incurre al despedir a un trabajador sin causa de caducidad de contrato según la ley. Los casos específicos son:
    \begin{itemize}
        \item \textbf{Desahucio}: Pago de un mes de sueldo al trabajador despedido sin previo aviso de al menos un mes. En obras transitorias como la construcción, este pago no se aplica.
        
        \item \textbf{Indemnización por años de servicio}: Pago de un mes de sueldo por cada año trabajado, con un tope máximo según la ley. En la construcción, generalmente no se paga en obras transitorias.
        
        \item \textbf{Pago proporcional por vacaciones}: Compensación al trabajador despedido antes de tomar sus vacaciones. Se le paga una cantidad proporcional a los días que le corresponderían según el tiempo trabajado.
    
        \item \textbf{Feriado anual (vacaciones)}:
        \begin{equation}
        \text{Feriado anual} = 21 \, \text{días} \times 8 \, \text{horas} \times 1.100 = 184.800 \, \text{pesos}
        \end{equation}
        \begin{equation}
        \text{Feriado anual (\%)} = \frac{184.800}{3.106.400} = 5,9\%
        \end{equation}
        
        \item \textbf{Causas climáticas}:
        \begin{equation}
        \text{Clima} = 10 \, \text{días} \times 8 \, \text{horas} \times 1.100 = 123.200 \, \text{pesos}
        \end{equation}
        
        \item \textbf{Días festivos}:
        \begin{equation}
        \text{Festivos} = 14 \, \text{días} \times 8 \, \text{horas} \times 1.100 = 88.000 \, \text{pesos}
        \end{equation}
        \begin{equation}
        \text{Festivos (\%)} = \frac{88.000}{3.106.400} = 4,0\%
        \end{equation}
        
        \item \textbf{Aguinaldos}:
        \begin{equation}
        \text{Aguinaldos} = 2 \times 25.000 = 50.000 \, \text{pesos}
        \end{equation}
        \begin{equation}
        \text{Aguinaldos (\%)} = \frac{50.000}{3.106.400} = 1,6\%
        \end{equation}
    \end{itemize}
\end{itemize}

\begin{figure}[H]
    \centering
    \includegraphics[width=0.7\textwidth]{FOTOS/indem.png}
    \label{fig:costo_mo}
\end{figure}

\subsubsection{Costo base de los materiales}

El costo de los materiales se basa en una cotización adecuada de los materiales a utilizar en la obra, diferenciada por tipo y considerando al proveedor más conveniente. El precio debe ser puesto en obra y puede verse afectado por varios factores:
\begin{itemize}
    \item \textbf{Formas de pago}
    \item \textbf{Volúmenes de compra}
    \item \textbf{Calidad de los materiales}
    \item \textbf{Ofertas del momento}
    \item \textbf{Otros factores}
\end{itemize}

Al analizar el costo de materiales, es recomendable incluir posibles pérdidas debido a:

\begin{itemize}
    \item Robos
    \item Mala utilización
    \item Mal almacenamiento
    \item Mal transporte
    \item Otras pérdidas significativas
\end{itemize}

\subsubsection{Costo base de los equipos}

En este rubro se incluyen herramientas (martillos, palas, carretillas, etc.), útiles (escaleras, andamios, etc.) y maquinarias (grúas, vibradores, etc.). En muchas empresas, el costo de herramientas y útiles se carga a gastos generales, mientras que para las maquinarias existen tres opciones:

\begin{itemize}
    \item \textbf{Equipos arrendados}: Se considera una tasa de arriendo, teniendo en cuenta qué costos están incluidos. Si ciertos costos como operador, mantención o accesorios no están incluidos, deben agregarse para estimar el costo real de operación.
    
    \item \textbf{Equipos con leasing}: Utilizan un instrumento financiero que implica un arriendo con compromiso de compra. El costo mensual es generalmente superior al de un arriendo tradicional, pero ofrece beneficios tributarios. Al final del período de leasing, se puede adquirir el equipo con una cuota adicional.
    
    \item \textbf{Equipos propios}: Requiere determinar los costos de depreciación, posesión y operación mediante algún método específico, que se desarrollará en el capítulo correspondiente.
\end{itemize}

\subsection{Precio unitario}

Conocidos los costos de los componentes principales de cada partida, se debe justificar su precio unitario. Para calcular estos precios se determina el costo total de la partida y se divide por la cantidad de unidades respectivas, considerando economías o deseconomías por volumen de obra.

Por ejemplo, el costo unitario de 100.000 m³ de hormigón es menor que el de 10 m³ debido a variaciones en precios de insumos y prorrateo de equipos. Otra opción es estimar el precio unitario de cada partida independientemente de la cubicación, considerando sólo materiales, equipos y personal necesarios para la unidad (e.g., m², ml, gl, UN). También es posible una combinación de ambos métodos para diferentes partes de la obra.

\begin{figure}[H]
    \centering
    \includegraphics[width=0.7\textwidth]{FOTOS/pu.png}
    \label{fig:precio_unitario}
\end{figure}

\subsection{Gastos}

\subsubsection{Gastos generales}

El cálculo de los gastos generales se divide en tres grupos principales:

\begin{itemize}
    \item \textbf{Personal}: Incluye gastos de personal no directamente involucrado en la ejecución de la obra.
    \begin{itemize}
        \item \textbf{Personal administrativo y de apoyo}: Profesionales, administrativos, laboratoristas, supervisores, jefes de turno, mecánicos, eléctricos, etc.
        \item \textbf{Asesores}: Auditores, calculistas, abogados, medio ambiente, seguridad.
        \item \textbf{Entretenciones}: Fiestas de los tijerales, club deportivo.
    \end{itemize}
    
    \item \textbf{Instalaciones}: Incluye todo lo relacionado con la operación de la obra.
    \begin{itemize}
        \item \textbf{Instalaciones de faena}: Oficinas, servicios higiénicos, talleres, bodegas, plantas, galpones, laboratorios.
        \item \textbf{Movilización}: Vehículos para personal, ambulancias, camionetas.
        \item \textbf{Viajes y visitas}: Costos de visitas y viajes de la empresa.
        \item \textbf{Gastos de oficina de obra}: Papelería, fotocopias, planos, correspondencia.
        \item \textbf{Gastos de servicios}: Agua, luz, telefonía, internet.
        \item \textbf{Policlínico}: Materiales médicos.
        \item \textbf{Seguridad e higiene}: Elementos de seguridad, señalización.
    \end{itemize}
    
    \item \textbf{Equipamiento}: Vehículos, fletes, equipos de laboratorio, computación y comunicaciones. Si no están en el precio unitario, se deben incluir aquí.
    \begin{itemize}
        \item \textbf{Fletes}: Ida y retorno de equipos, materiales y mudanzas de personal.
        \item \textbf{Control de calidad}: Laboratorio externo.
        \item \textbf{Topografía}: Arriendo y mantenimiento de equipos, estacas, puntos de referencia.
        \item \textbf{Comunicaciones}: Radio, teléfono, internet.
        \item \textbf{Despachos}: Correspondencia y encomiendas.
    \end{itemize}
\end{itemize}

\subsubsection{Gastos generales indirectos}

Los gastos generales indirectos son aquellos que se generan por la realización del proyecto, pero no directamente por su construcción, como la oficina central, costos financieros, varios e imprevistos.

\begin{itemize}
    \item \textbf{Oficina Central}: El costo de la oficina central de la empresa debe distribuirse entre todas las obras activas. El cálculo se realiza estimando un porcentaje del costo directo, o bien, se pueden considerar por separado los siguientes costos:
    \begin{itemize}
        \item \textbf{Gastos de oficina general}: Contabilidad, administración, etc.
        \item \textbf{Gastos de representación}: Publicidad, atención a personalidades, eventos.
    \end{itemize}
    
    \item \textbf{Costo Financiero}: Se calcula a partir del flujo de caja neto del proyecto, estimando la diferencia entre los gastos programados y los ingresos esperados. La estimación se basa en el programa mensual de inversiones y el flujo mensual de gastos generales. La tasa de interés utilizada depende del banco o indicador financiero de referencia.
    
    \item \textbf{Costos Indirectos Varios}: Gastos asociados a efectos del proyecto, asesorías y aspectos legales. Algunos ejemplos son:
    \begin{itemize}
        \item \textbf{Gastos de propuesta}: Costos de estudio, viajes, etc.
        \item \textbf{Garantías}: Boletas de seriedad, garantía de cumplimiento y correcta ejecución.
        \item \textbf{Costos de notaría}: Escrituras de contrato, protocolizaciones.
        \item \textbf{Derechos y permisos}: Permisos municipales, exploración de canteras, permisos de agua.
        \item \textbf{Seguros}: Responsabilidad civil, vehículos, vida de empleados, daños a terceros, incendio, equipos. Existe un seguro a todo riesgo de construcción que cubre al personal, equipos y obra desde el inicio hasta la finalización del proyecto.
    \end{itemize}
\end{itemize}

\subsection{Presentación de un presupuesto}

La presentación de presupuestos es similar para cada tipo de contrato, pero la interpretación varía según el tipo:

\begin{itemize}
    \item \textbf{Administración delegada}: El presupuesto es una buena estimación del costo, ya que el dueño cubrirá todos los gastos incurridos por el contratista. Es esencial definir los honorarios del contratista.
    
    \item \textbf{Contratos a serie de precios unitarios}: Lo importante son las partidas y sus precios unitarios, ya que las cantidades son estimaciones. El mandante pagará al contratista por las cantidades efectivamente ejecutadas a los precios unitarios acordados.
    
    \item \textbf{Contratos de suma alzada}: La presentación del presupuesto es referencial, ya que el dueño pagará el monto total acordado independientemente de las estimaciones.
\end{itemize}

\begin{figure}[H]
    \centering
    \includegraphics[width=0.8\textwidth]{FOTOS/ej_pu.png}
    \label{fig:presupuesto}
\end{figure}

Para la presentación del presupuesto u oferta final, la empresa constructora debe distribuir todos los gastos distintos a los costos directos en las partidas consideradas ítems de pago. Existen varias opciones para hacerlo, como la distribución porcentual. Además, la oferta final puede variar según la estrategia adoptada para la propuesta, ajustándose según el interés del contratista en la obra o la carga de trabajo existente.

El uso de tecnología computacional facilita el estudio de presupuestos, acelerando el proceso. Sin embargo, la responsabilidad final recae en quien usa el programa. Las empresas pueden diseñar sus propios programas o adquirir alguno disponible en el mercado. En Chile, algunos programas comunes para el estudio de presupuestos son: Notrasnoches (plataforma ONDAC), Presto y Unysoft.

\subsubsection{Reajuste de presupuesto}

Las obras de construcción suelen tener períodos de ejecución prolongados, lo que hace necesario considerar reajustes de precios para actualizar los montos. Estos reajustes se aplican en los pagos parciales o al término de la obra. Las formas de considerar los reajustes incluyen:

\begin{itemize}
    \item \textbf{Índice Reajustable}: Confeccionar el presupuesto sobre la base de un índice que pueda ser reajustado, como la UF.
    
    \item \textbf{Polinomios de Reajustes}: Utilizar polinomios que combinan diferentes índices en proporciones adecuadas para reflejar la variación real del costo de la obra.
    
    \item Ejemplo de Reajuste:
    \begin{align}
    \text{Reajuste} &= 0.20 \times \text{(variación del costo de la mano de obra)} \nonumber \\
    &\quad + 0.40 \times \text{(variación del precio de los materiales)} \nonumber \\
    &\quad + 0.10 \times \text{(variación del dólar)} \nonumber \\
    &\quad + 0.30 \times \text{(variación de la unidad de fomento)}
    \end{align}  
\end{itemize}

En contratos con cláusulas de reajuste de precios, es recomendable considerar la posible diferencia entre el índice contractual de reajuste y la variación real del costo. Si se prevé una diferencia significativa, se sugiere ajustar el presupuesto en consecuencia. Este aspecto es especialmente importante en proyectos sin reajustes.

El MOP ofrece tres opciones para reajustes: sin reajustes, con reajuste de IPC, y la opción polinómica.

Si el IPC es reemplazado por otro índice determinado por el INE o la autoridad competente, el reajuste se calculará con base en el nuevo índice. El pago del reajuste se realizará junto con el estado de pago de obra correspondiente.

Cuando se realicen pagos de reajuste en los contratos, su valor se ajustará automáticamente, ya sea aumentando o disminuyendo, sin esperar la resolución final. En caso de discrepancias en estas resoluciones, las retenciones y garantías del contrato cubrirán posibles excesos pagados. Los precios unitarios acordados en el contrato se mantendrán vigentes para otros efectos, excepto para temas de presupuesto y pago.

\subsection{Sistemas de Pago}

En Chile, la modalidad más común en obras es la de **Estados de Pago (EP)**, que refleja el avance físico de la obra en términos monetarios. Considera elementos como avance físico, retenciones, descuentos y devoluciones. El proceso de pago incluye:

\begin{itemize}
    \item \textbf{Detalle de Estado de Pago}:
    \begin{itemize}
        \item Antecedentes de obra
        \item Número correlativo y período
        \item Partidas contempladas
        \item Monto contratado (unidad, cantidad, P.U.)
        \item Obra realizada hasta la fecha (unidad, cantidad, P.U., total)
        \item Valor realizado hasta el estado de pago anterior (total)
        \item Valor del presente estado de pago (diferencia entre el actual y el anterior EP)
    \end{itemize}
    
    \item \textbf{Carátula del Estado de Pago}:
    \begin{itemize}
        \item Antecedentes de obra
        \item Número de estado de pago y período que abarca
        \item Valor realizado a la fecha
        \item Valor realizado hasta el estado de pago anterior
        \item Retenciones
        \item Devoluciones
        \item Descuentos
        \item Líquido a pagar
    \end{itemize}
    
    \item \textbf{Factura}
\end{itemize}

El estado de pago debe ser revisado y aprobado por la inspección antes de ser pagado por el dueño. Los pagos pueden realizarse en diferentes períodos de tiempo, de acuerdo con las condiciones estipuladas en el contrato.

\begin{figure}[H]
    \centering
    \includegraphics[width=0.8\textwidth]{FOTOS/sist_pag.png}
    \label{fig:estado_pago}
\end{figure}




