\part{Capítulo 6}
\section{Clasificación de estructuras}
\begin{itemize}
    \item Clase A: Edificios de acero con losas de homigón armado o entrepisos de acero
    \item Clase B: Edificios de hormigón con losas de hormigón o estructuras mixtas de hormigón y acero
    \item Clase C: Edificios de albañilería de ladrillo confinado entre cadenas y pilares de hormigón armado
    \item Clase D: Edificios de albañileria de bloques o piedras confinado entre cadenas y pilares de hormigón armado
    \item Clase E: Edificios de estructuras de madera, con entrepisos de madera
    \item Clase F: Edificios de estructuras de adobe, con entrepisos de madera
    \item Clase G: Edificios de estructuras prefabricadas de madera, acero u hormigón
    \item Clase H: Edificios de estructuras prefabricadas de madera
    \item Clase I: Edificios de estructuras prefabricadas o paneles de hormigón liviano, fibrocemento o poliestireno expandido
\end{itemize}

\section{Componenetes de una edificación}
\begin{itemize}
    \item Infraestructura o Fundación
    \item Supraestructura o Cuerpo
    \item Techumbre
    \item Terminaciones
    \item Instalaciones
\end{itemize}

\subsection{Urbanización}
\begin{itemize}
    \item Trazados viales y urbanos
    \item Áreas verdes y equipamientos
    \item Iluminación y ventilación
    \item Dotación de servicios sanitarios, energéticos y de comunicación
\end{itemize}
\subsection{Instalación de Faenas}
Instalaciónes provisorias de apoyo a la construcción de la obra
\begin{itemize}
    \item Cierro provisorio
    \item Portería e ingresos
    \item Instalaciones Administrativas
    \item Instalaciones de Personal
    \item servicios
    \item Servicios Básicos
    \item Caminos de acceso y de circulación
\end{itemize}
\subsection{Topografía}
Es la medición y representación de la superficie terrestre
\begin{itemize}
    \item Mapas: (1:250.000 a 1:2.500.000)
    \item Cartas: (1:25.000 a 1:100.000)
    \item Planos: (1:10 a 1:1.000)
    \item Se usa coordenadas UTM
    \item levantamiento topográfico
    \begin{itemize}
        \item Obtención de datos
        \item Procesamiento de datos
        \item Confección de planos
    \end{itemize}
    \item Replanteo para una verificación y control
\end{itemize}