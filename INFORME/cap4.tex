\part{Capítulo 4: Contratos y propuestas en proyectos de Construcción}

Es un convenio entre el que construye y el dueño o mandante que financia, y fija sus objetivos
de acuerdo con sus necesidades y posibilidades. El propósito de un contrato de construcción
es definir derechos, obligaciones y responsabilidades de cada una de las partes involucradas.
\\ \\
El propietario puede designar una inspección para controlar la obra. Es como intermediario entre
contratista y mandante. Hay contratos que le adjutican esta responsabilidad a un "Árbitro".

\begin{itemize}
    \item Modalidades de Contratos de construcción
    \begin{itemize}
        \item Construir para sí: Usa sus propios recursos para la ejecución de proyectos arriesgándose a la no aceptación de este por parte del mercado inmobiliario.
        \item Construir por terceros: Obra es financiada por el mandante, y el contratista ejecuta. Pueden tener las siguientes relaciones.
        \begin{itemize}
            \item Mandante entrega el proyecto y lo financia, contratista lo ejecuta.
            \item Mandante financia la obra, solicita el diseño y la ejecución al contratista.
            \item Contratista diseña, ejecuta y financia la obra y entrega la obra terminada al mandante en un precio previamente convenido (contrato llave en mano).
        \end{itemize}
    \end{itemize}
    \item Tipos de Contratos
    \begin{itemize}
        \item Contrato de suma alzada: Contratista realiza toda la obra a un precio fijo (propuesto por él después de estudiar el proyecto y aceptado por el mandante). Máximo riesgo es del contratista.
        \begin{itemize}
            \item Proyecto tiene que estar 100\% definido.
            \item Dueño escoge la mejor oferta.
            \item Los cambios son casi imposibles por parte del mandante, debido al contrato de adjudicación.
            \item Contratista debe hacer un análisis de costos precisos para dar la oferta.
        \end{itemize}
        \item Contrato de precios unitarios: Se establecen precios unitarios para cada partida de la obra, y se paga por la cantidad de trabajo realizado. Se paga por lo que se hace. El riesgo es compartido entre mandante y contratista. Es competitivo.
        \begin{itemize}
            \item Se puede ofertar sin tener el proyecto definido.
            \item Permite al dueño saber exactamente cuánto va a invertir en la obra.
            \item Contratista deberá realizar un estudio de costos preciso.
        \end{itemize}
        \item Contrato de administración delegada: Contratista se encarga de la administración de la obra, y el mandante paga los costos de la obra y un porcentaje adicional por la administración (honorarios). El riesgo es del mandante. Se recomienda como opción de emergencia y sin competencia, cuando se tiene completo el proyecto y se debe cumplir en un plazo corto. Se requiere confianza e inspecciones constantes.
        \begin{itemize}
            \item Dueño no conoce el presupuesto final.
            \item Contratista no corre riesgo con ganancias.
            \item Contratista puede encarecer la obra.
            \item Si los honorarios son fijos, contratista se motiva a terminar antes.
            \item Si honorarios tienen incentivo por horario/plazo, contratista se motiva a cumplir.
        \end{itemize}
    \end{itemize}
    \item Condiciones previas al llamado de una propuesta.
    \begin{itemize}
        \item Mandante debe tener claro lo que se quiere construir, el costo, el financiamiento y adicionales.
        \item Mandante debe informar al proyectista el costo aproximado de la obra.
        \item Mandante debe avisar al proyectista el tipo de contrato que se concretará.
        \item Incluir método constructivo en el diseño del proyecto.
        \item Se deben elaborar las bases administrativas por las que se regirá el contrato.
        \item Mandante podría encargar un estudio de presupuesto e inversión oficial de la obra. Se suele saltar esto.
        \item Establecer clara y rígidamente el sistema de pago que se implantará y la fuente de financiamiento de la obra.
        \item Elaborar el proyecto a cabalidad y en lo posible concertar una o más reuniones con todos los proyectistas participantes.
        \item Existen propuestas públicas (todos los que cumplan los requisitos) y privadas (aquellos invitados).
        \item OJO: El Estado está obligado por carta fundamental a llamar licitaciones públicas en primera instancia.
        \item Registro y pre-clasificación de contratistas:
        \begin{itemize}
            \item Clasifican a las empresas por especialidad, tamaño, experiencia, capital, etc.
            \item Antes del llamado de licitación, se debe preseleccionar número de proponentes y los requisitos mínimos que debe satisfacer el contratista.
            \item A todos se les entrega la misma información, calendario estricto del proceso de licitación en lo que se refiere a retiro de bases y antecedentes, plazo para consultas, plazo para respuestas, fecha de apertura o de recepción de ofertas y fecha de adjudicación de la obra.
            \item Establecer plazo máximo para ofertar.
        \end{itemize}
        \item Documentos principales de una propuesta:
        \begin{enumerate}
            \item Instrucciones a los proponentes.
            \item Bases generales.
            \item Propuesta o formularios de la propuesta.
            \item Bases especiales.
            \item Especificaciones técnicas.
            \item Planos del proyecto.
            \item Documentos de referencia.
            \item Serie de preguntas y respuestas.
            \item Apéndices.
            \item Antecedentes técnicos complementarios sobre el terreno o sus accesos.
        \end{enumerate}
        \item Evaluación y adjudicación de una propuesta.
        \begin{itemize}
            \item Propuestas son recibidas y abiertas por una comisión designada por el propietario, durante una reunión donde se leen algunos datos relevantes y se registran en un acta de apertura. Este proceso puede realizarse de manera electrónica, garantizando transparencia para todos los oferentes y el público. Un ejemplo de esto es el portal mercadopúblico.cl. Se emiten dos actas: una de apertura y otra de evaluación de ofertas, que se crean tras la apertura técnica y económica. Durante la evaluación, se realiza un análisis comparativo de las ofertas técnicas y económicas.
            \item La nota final de evaluación técnica se calcula de la siguiente manera:
            \begin{equation}
                NFt = \sum_{i=1}^{n} (X_i \times Y_i)
            \end{equation}
            Con:
            \begin{equation}
                \sum_{i=1}^{n} Y_i = 1 
            \end{equation}
            Donde:
            \begin{itemize}
                \item $X_1$ = Experiencia y antecedentes de la empresa
                \item $X_2$ = Tipo de organización y metodología que se ofrecen
                \item $X_3$ = Equipo de trabajo ofrecido
                \item $X_4$ = Seriedad
                \item $X_5$ = Capacidad económica
                \item $X_6$ = Capacidad técnica
            \end{itemize}
            \item Luego se hace la evaluación económica sobre la base del valor de la oferta (NFe). Se obtiene la nota final (NF) usando la ponderación para cada evaluación.
            \begin{equation}
                NF = NFt \times P_t + NFe \times P_e
            \end{equation}
        \end{itemize}
    \end{itemize}
\end{itemize}
 